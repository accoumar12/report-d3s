\chapter*{Conclusion}
\addcontentsline{toc}{chapter}{Conclusion}
\markboth{Conclusion}{Conclusion}
\label{sec:conclusion}

This work presents a novel approach to addressing the challenge of 3D CAD design similarity in industrial settings, where large repositories of designs often lack consistent labeling and organization. We developed a pipeline that combines innovative triplet generation techniques, a user-friendly labeling application, and state-of-the-art deep learning models to create an effective similarity model for 3D CAD designs.
Our key contributions include:


\begin{itemize}
    \item A method for generating unlabeled triplets from an extensive dataset of 3D designs, balancing diversity, labelability, and difficulty.
    \item A "Tinder-like" application that facilitates efficient human labeling of triplets, incorporating both default and canonical views to aid in comparison.
    \item An iterative process of model training and triplet generation, allowing the model to continuously improve by learning from its previous misconceptions.
    \item Experimentation with both graph-based (GNN) and transformer-based models, ultimately finding superior performance with the transformer-based approach.
\end{itemize}

The experiments conducted demonstrate the effectiveness of our approach. The graph-based model using EdgeConv layers showed promising initial results, achieving high accuracy on the test set. However, it struggled to maintain performance on the original triplet order metric over extended training. The transition to a transformer-based model, specifically using the Openshape encoder architecture, yielded significantly improved results. This model not only maintained high performance across all metrics but also showed continuous improvement in triplet ordering and rotation invariance as training progressed.

One of the main advantages of our approach is its capacity to learn meaningful representations from a relatively small set of labeled triplets. With only 20,000 triplets, we successfully trained a GNN model that exceeded our initial expectations and demonstrated strong performance across multiple metrics, although it falls short of the larger transformer model. The pre-trained version already shows superior results, and fine-tuning further enhances its performance.

However, our work also has some limitations:

\begin{itemize}
    \item The reliance on human labeling, while mitigated by our efficient application, still introduces a bottleneck in scaling to very large datasets.
    \item The potential for bias in the triplet generation process, which may not fully capture all nuances of design similarity.
\end{itemize}


In conclusion, this work represents a significant step forward in automating the classification and retrieval of 3D CAD designs. By leveraging human expertise through targeted labeling and state-of-the-art deep learning techniques, we have created a robust similarity model that can greatly enhance the efficiency of design reuse and information retrieval in industrial settings. 

%%% Local Variables: 
%%% mode: latex
%%% TeX-master: "isae-report-template"
%%% End: 

